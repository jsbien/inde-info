\documentclass[12]{mwart}
\usepackage{polyglossia}
\setdefaultlanguage{polish}

\usepackage{enumitem}
\usepackage{draftwatermark}

\usepackage{xltxtra}

\setmainfont[Mapping=tex-text]{TeX Gyre Termes}
\setsansfont[Mapping=tex-text]{TeX Gyre Adventor}
%\setmainfont{TeXGyreTermes}
%\setmainfont{DejaVu Serif}
%\setmainfont{Bitstream Vera Serif}
%\setmonofont{TeX Gyre Cursor}
\setmonofont{DejaVu Sans Mono}
\usepackage{draftwatermark}

% \usepackage{bibentry,natbib}

\usepackage{graphicx}

\usepackage{hyperref}

\usepackage{soul}

\usepackage{relsize}

\usepackage[style=authoryear,natbib=true]{biblatex}
%\addbibresource{JSB2013.bib,typografia.bib}
\addbibresource{4JSB2014.bib}
\AtEveryBibitem{\clearfield{note}}



\newcommand{\program}[1]{\textsf{#1}}

\title{Tryb korekty w \texttt{djview4poliqarp}}
\author{Janusz S. Bień}

\date{24.03.2014, \ldots, 26.05.2018}
%  8.04.2014, ??.10.2014}

\begin{document}
\maketitle
% \pagestyle{empty}

% no math
\catcode`\&=12
\catcode`\_=12

\begin{quote}
  Tekst na otwartej licencji Creative Commons Uznanie Autorstwa,
  źródła dostępne w repozytorium
  \url{https://bitbucket.org/jsbien/linde-info}.
\end{quote}

\section{Wstęp}
\label{sec:wstp}

Tryb korekty polega na kolejnym wyświetlaniu haseł z indeksu, przy
czym dla każdego hasła użytkownik ma 4 możliwości:
\begin{itemize}
\item zaakceptować,
\item odrzucić,
\item poprawić,
\item pominąć.
\end{itemize}

Wybór możliwości powinien być prosty i odbywać przez pojedyncze
naciśniecie klawisza lub ich kombinacji. Byłoby dobrze, gdyby te
funkcje klawiszy były konfigurowalne.

Z wyjątkiem poprawiania gabarytów wszystkie operacje powinne dać się
sprawnie wykonywać z klawiatury.

Pożądana jakaś choćby prymitywna operacja cofania przynajmniej jednej
ostatniej operacji.

Dla odróżnienia haseł zaakceptowanych, odrzuconych itp. tymczasowo
możemy stosować obecną konwencję oznaczania haseł ukrytych komentarzem
zaczynającym się od \texttt{!}. Docelowo chyba dla statusu hasła
trzeba będzie wprowadzić dodatkowe pole.

\section{Wyświetlanie hasła}
\label{sec:wywietlanie-hasa}

Sposób wyświetlania hasła powinien być konfigurowalny i obejmować co
najmniej dwie możliwości.

W najprostszym wypadku wystarczy standardowe wyświetlanie zaznaczenia
--- będzie to całkowicie wystarczające dla indeksów przeznaczonych do
weryfikacji rozpoznania układu strony.

Drugi sposób wyswietlania hasła to zaznaczenie uzupełnione o tekst
ukryty. W tej chwili wymaga to dodatkowego kliknięcia i naciśnięcia
Shift, poza tym zakres wyświetlanego tekstu nie ma związku z
zaznaczniem. Tekst wyświetlany w trybie korekty powinien dokładnie
odpowiadać zaznaczeniu.

Do rozważeni jest również trzeci sposób polegający na wyświetlaniu
również komentarza hasła, ale na razie nie widać dla niego konkretnych
zastosowań.

\section{Akceptacja hasła}
\label{sec:akceptacja-hasa}

Akceptacja hasła polega na jego ukryciu i przejściu do następnego
hasła. Ukrycie odbywa się przez wstawienie do komentarza na jego
początku znaków \texttt{!+!}.

Proponowany klawisz: spacja lub Enter.

\section{Odrzucenie hasła}
\label{sec:odrzucenie-hasa}

Funkcja będzie chyba używana rzadko. Polega na opatrzeniu hasła
odpowiednim komentarzem, ukryciu go i przejściu do następnego hasła.
Ukrycie i oznaczeni odbywa się przez wstawienie do komentarza na jego
początku znaków \texttt{!-!}.

Proponowany klawisz: może Backspace?

\section{Poprawianie hasła}
\label{sec:poprawianie-hasa}

Nie jest jasne, czy potrzebne będzie jednoczesne poprawianie gabarytów
i treści hasła, dlatego tego przypadku nie omawiam.

\subsection{Poprawianie gabarytów hasła}
\label{sec:popr-gabaryt-hasa}

Przypadek ten dotyczy przede wszystkim haseł reprezentujących różne
elementy układu strony, np. akapity.

Za pomocą myszy poprawiamy gabaryty i akceptujemy hasło w sposób
opisany wyżej. To jest jedyny przypadek, kiedy do obsługi programu
jest niezbędna mysz.

\subsection{Poprawianie treści hasła}
\label{sec:popr-treci-hasa}

Przypadek ten dotyczy przede wszystkim indeksów, w których hasło i
domyślna treść hasła --- formalnie komentarz --- to segmenty uzyskane
w wyniku OCR.

Po wybraniu tej opcji powinno się otworzyć okienko do edycji
komentarza (inne pola chyba powinne być chronione). Okno nie powinno
przesłaniać skanu z oryginałem - może otwierać je w lewym panelu pod
indeksem?

Po akceptacji hasło powinno zostać ukryte.

Proponowane klawisze: może Ctrl-spacja lub Ctrl-Enter?

\subsection{Pomijanie hasła}
\label{sec:pomijanie-hasa}

Stosowane np. gdy hasło wymaga głębszego zastanowienia. Następuje
przejście do następnego hasła, ale hasło bieżąco pozostaje
niezmienione --- w szczególności nie zostaje ukryte.

Proponowane klawisze: może Alt-spacja lub Alt-Enter?


\section{Zastosowanie}
\label{sec:zastosowanie}

Indeksy dla programu zostaną wygenerowane z bazy danych informacji
typograficznych.

Część użytkowników będzie ,,przygodnych'' --- np. studenci
przeprowadzający korektę na zaliczenie zajęć.

\end{document}

%%% Local Variables:
%%% mode: latex
%%% TeX-master: t
%%% TeX-engine: xetex
%%% TeX-PDF-mode: t
%%% coding: utf-8
%%% End:
