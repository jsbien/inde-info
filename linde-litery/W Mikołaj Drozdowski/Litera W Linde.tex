%http://teksty.klf.uw.edu.pl/30/4/LindeIIGP%2B6i.djvu%
%http://korpusy.klf.uw.edu.pl/djvus/linde/index.djvu?djvuopts=&page=3877&zoom=100&showposition=0.51,0.66&djvuopts=&highlight=117,360,2103,587
%http://korpusy.klf.uw.edu.pl/djvus/linde/index.djvu?djvuopts=&page=3878&zoom=100&showposition=0.5,0.32&djvuopts=&highlight=174,2228,1046,624


W, |{litera 22ga  abecadła Polskiego, składa się z dwóch vv
łacińskich.} Kras. Zb. |{2, 562.} Caremus simplici v,
quod apud Germanos cum litera f unum eundemque,
sonum obtinuit}, e, |{gr.}, Beftung, feft. Rossici w ex-
primunt figura simili nostrae literae b: B, et ter-
tium ipsi in ordine alphabetico locum assignat, statim
post 6, quae sonum habet nostri b. -- |{U nas zaś w
jest dwojakie, n. p. w radzie; i z kreską n.p. ulów'.}
Januszowski uważa, |{⸗że się w za nas wzięło z Nie-
mieckiego; jeno już lężej u nas idzie, jak u Niemców⸗.}
Now. Ch. Sub initio densius quidem haec litera sonat,
quam f, ut |{wada}, |{wiem}, |{wołam}, |{wbijam}, sed in fine
f sono non discernitur, ul |{traw a trawa} et |{traf}; sic
traw' a trawić et trafic. In medio utrumque
sonum habet, ut: 1) |{dawno} , |{łowię} et 2) |{owcug}, |{ło-
wczy}, |{przeciwko}, quae possent etiam f pro w habere,
ut 1is:fa, plalfy, uboslfo, slfolin. Cn. Th. 1212., Haec
est ratio, cur f, quae litera nequaquam Slavonica est,
saepenumero , praesertim in veteribus libris, locum
obtinuit literae w, et vice versa; e. |{gr}. 'ukwasnieć.
Sekl. 67, krotofilć ⸗ krotochwilić, ufalić. Biel. Św.
|{265;} -|{ufały}, |{zufaly}, a verbo ufac' inde |{zuchwały}-
Caeterum w proximam habet affinitatem cum litera b,
quae ex innumera commutatione earum clucet: |{boj},
|{bojować}, |{wojować} , |{wojna;} |{wrócić}, |{odwrócić}, |{odwrot},
|{obrot}; |{powiesić}, |{obiesić [pro obwrót, obwiesić 2}. -
W seapissime apud. Polonos in formam spiritus asperi, 
voycabula incipit, quae apud Rossiios, Bohemos etc.,
a vocali, sine w incipiunt, praesertim ante -- ą-- ę,
e. |{gr}. |{węzeł, Boh}. uzel, |{Ross}, yзeкъ; wązki, |{Boh}. auzky;
|{Ross.} yзкiй; wtorek, |{Boh.} autorek; |{Ross.} вторкикъ.
W.... abrewiacya ⸗ Wielki, n. p. W. X. L. ⸗ Wielkie Księ-
stwo Litewskie —- Wielce: MWMP ⸗ mój Wielce Mości-
wy Panie;-- W. 0. ⸗ Wielebny Ojciec; Wc ⸗ Wac Pan;
WMści Dobr. ⸗ Wielmożny Mości Dobrodzieju. WWPD. ⸗
Wielmożny Wac Pan Dobrodziej. -- |{Simililer}: JW. ⸗
Jaśnie Wielmożny. n. p. Pracowali koło edukacyi naro-
dowéj J. J. W. W. Chreptowicz, Podkanclerzy W. X. L.
Potocki, Pisarz W. W. X. L. etc. |{Zab}. |{16, 172.} — W.
K. M. ⸗ Wasza Królewska |{M}ość.

\endinput

W, litera ?.an abecadlo Pols/ciego, składa się z dwóch vv
łacińskich. Kras. Zb. 2, 562. Caremus simplici v,
quod apud Germanos cum litera funum eundemque,
sonum obtinuit, o, gr., Vestung, fejl. Rossicź w ex-
primunt figura simili nostrae literae b: B, et ter-
tium ipsi in ordine alphabetico locum assignat, statim
post 6, quae sonum habet nostri b. -- U nas zas' w
jest dwoja/rie, n. p. w radzie; i z kreską 11. p. ulo'w'.
Januszowski uważa, «źe się w za nas wzięło' z Nie-
miec/siego; jeno juź lęźej u nas idzie, jak u Niemców.
Now. Ch. Sub initio densius quidem haec litera sonat,
quam f, ut wada, wiem, wolam, wbijam, sed in ñne 
f sono non discernitur, ut traw a trawa et traf; sic
traw' a trawie' ct traf' a tra/iel ln mediu utrumque
sonum habet, ut: i) dazuno , łowię et 2) owcug, co-
wezy, przeciwko, quae possent etiam f pro w habere,
ut 1is.'fa, platfy, ubostfo, slfolin. Cn. Th. V1212., Haecest
ratio, cur l`, quae litera nequaquam Slavonica est,
saepenumero , pracsertim in veteribus libris, locum
obtinuit literce w, et vice versa; e.„gr. 'uk/as'niec'.
Sekl. 67, krotofilźc'- krotochwilic', ufalz'ć. Biel. Sw.
265; --ufa-ly, zufaly, a verbo ufac' inde zuchwały-
Caeterum w proximam habet aflinitatem cum litera b,
quae cx innumera commutatione earum clucet': boj,
bojowac', -wojować , wojna; wrócic', odwrócić, odwrot,
obrot; powiesić, obiesić [pro obwrót, obwiesic' -
W seapissime apud. Polonos in formam spiritus asperi,
voycabula incipit, quae apud àlossiqs, Bohemos etc.,
a vocali, sine w incipiunt, praesertim ante -- ą-..- ę,
e. gr. węzeł, Boh. uzel, Boss, yserb; wązki, Bobi auzky;
Boss. yasm.; wtorek, Boh. autorek; Boss. crcpunnz.
W.... abrewiacya - Wielki, n. p. W. X. I...: Wielkie' Księ-
stwo Litewskie «-.—- Wielce: MWMPs'mój Wielce Mości-
wy Panie;--'- W. 0.2` Wielebny Ojciec; Wc - Wac Pan;
WMści Dobr. - Wielmożny Mości Dobrodzieju. WWPD.-
Wielmoäny Wac Pan Dobrodziej. - Similźter: JW.;
Jaśnie Wielmożny. n.. p. Pracov'rali koło edukacyi naro-
'dowéj J.- J. W; -W. Chreptowicz, Podkanclerzy W. X. L.
Potocki, Pisarz W; W. X.. L.'etc. Zab. 16, 172. — W.
K. M.: Wasza Królewska Mość.

%%% Local Variables: 
%%% coding: utf-8
%%% mode: latex
%%% TeX-engine: xetex
%%% TeX-master: t
%%% TeX-PDF-mode: t
%%% End: 