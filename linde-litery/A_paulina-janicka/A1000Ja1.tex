kursywa = |{tekst}
goty @{tekst}

http://teksty.klf.uw.edu.pl/25/3/LindeIIGP%2B1i.djvu?djvuopts=&page=131&zoom=width&showposition=0.0,0.1&highlight=210,5400,131,193


A., Głoska najpierwsza wszystkich abecadeł, oprócz Etyop-

  skiego, w którym trzynasta. |{Kras}. |{Zb}. 1. 1.@{der erste Buch-
  stabe fast aller Alphabete} -`|{Phras}. Przedtem a b ledwo

  mówił jęcząc, Teraz |{r łacno} mówi, nie męcząc się. |{Jabł.

  Ez. A} 4. (postępek od łatwiejszego do trudniejszego).

  Bóg pożycia osnowę twego wywiódł, i dał tę, żeś rzekł
  o, a, mowę pierwszą z płaczem. |{Kulig. Her}. 33. (Carn.

  ar, as początek, origo.) Jam jest alpha i omega. 1. |{Leop.

  Apoc}. 1.8. (z Greck. a i z, początek i koniec).


Samogłoskę a wymawiamy trojako: 1. otwarto, n. p. rada.

  Nad takim a w dawnych książkach znajdziesz znamię pra-

  we: a; 2. mniej otwarto, n. p. wolá. Tego a u da-

  wnych nie'znaczono; teraz zaś nie otwarte a, lecz ści-

  śnione kreskować się zwykło. [Według Menińskiego Gram.

  Pol. Dant. 1649 str. 1 a ściśnione wymawiano jak fran-

  cuzkie au; dziś w języku pismiennym nie ma żadnej

  między ściśnionem a otwartem a różnicy, i kreskowania

  głoski téj zupełnie zaniechano. 1.| 3. ą, z kreseczką lub

  półmiesiączkiem u dołu, wymawia się, wypuszczając tro-


  chę tchu przez nos, i dlatego nazwane nosowe, n. p.

  są.


\endinput

A., Głoska najpierwsza wszystkich abecadeł, oprócz Etyop-

  skiego, wktórym trzynasta. Kras. Zb. l. l., der erste Buch-

  stabe fast aller Alphabete. -`- Phras. Przedtem' a `b ledwo

  mówił jęcząc, Teraz 7' łacno mówi, nie męcząc się. Jab?.

  Ez. A 4. (postępek od łatwiejszego do trudniejszego).

  Bóg pożycia osnowę twego wywiódł, _i dał tę, żeś rzekł
to,
  a, mowę pierwszą z płaczem." Kulig. Her. 55. (Carn.

  ar, as - początek, oriyo.) Jam jest alpha iomega. l. Leop.

  Apoc. l. 8. (z Greek.:a i z, początek i koniec).


Samogłoskę a' wymawiamy, trojako: l. otwarto, n. p. rada.

  Nad takim a w dawnych książkach znajdziesz znamię pra-

  we: á; mniéj otwarto, n. p. wolá.
Tego a u da-
  wnych nie'znaczono; teraz zaś nie otwarte a, lecz ści-

  śnione kreskować się zwykło. [Według Menińskiego_Gram.

  Pol. Dant. 1649 str. '1 a ściśnione wymawiano jak fran-

  cuzkie au; dziś w języku pismiennym nie ma żadnej

  między ściśnionem- a' otwartem a różnicy, i kreskowania

  głoski téj`zupełnie 	zaniechano. l.] 5. ą, z kreseczką lub

  półmiesiączkiem u dołu, wymawia się, wypuszczając tro-

  1000!!!!!!!!!!!!!!!!!!!!!!!!!!!!!!!!!!!!!!!!!!!!!!!!!

  chę tchu przez nos, i dlatego nazwane nosowe, n. p.

  są. Kopce. Gr. l. 25, Przed b i p brzmi ą jak om,

  n. p._ dąb, rząp; przed inszemi spółgłoskami jak on, n. p;

  zając. liass. Gr. W wyrazie zabáczą znajdują się wszys-

  tkie trzy - gatunki głoski a. Już _i Jan Kochanowski

  zamyślał o przeniesieniu kréski z otwartego a na ści-

  śnione, nawet owprowadzeniu nowéj jakiejsiś głoski za-

  miast a'. -- Nowy char. Drehfaches 5xiolnisches a: 1. offenes;

  sonst mit deni Accent, jetzt ohne Accent; 2. dumpferes, wird 

  jetzt acceninirt; 3. Eediillensn. wird am Ende des Wortes

  fast wie ong, vor b und p wie om, vor den übrigen Con-

  sonanten wie on ausgesprochen." -