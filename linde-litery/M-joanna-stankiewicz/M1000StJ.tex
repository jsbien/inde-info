% http://teksty.klf.uw.edu.pl/27/3/LindeIIGP%2B3i.djvu
% http://teksty.klf.uw.edu.pl/27/3/LindeIIGP+3i.djvu
% http://teksty.klf.uw.edu.pl/27/3/LindeIIGP+3i.djvu?djvuopts=&page=13&zoom=width&showposition=0.5,0.45&highlight=221,2654,2090,2112

|{Jedenasta litera Łacińskiego abecadła}; |{Grecy ją zowią} my,
|{w Hebrajskim języku} mem. Kras. Zb. 2, 79. J. Ko-
chanowski |{od pospolitego} m, |{n}. |{p}, |{do}m, |{różni drugie
kreskowane}, |{n}. |{p}. |{uskro}m'; |{na tym} Gornicki |{kładzie
daszek}. Now. Char.

§⃔. 1. M fini nominis additum, primam personam ver-
bi |{jestem} significat: |{mocny}m |{teraz} ⸗ |{mocny} jestem.
|{Trzeźwy}m|{ci}, |{żywe}m|{ci} vel |{żyw}-|{e}m|{ci}, id est, |{wżdy ży}-
|{wy} jestem. In his vero: |{byli}m |{tam}, |{slyszeli}m |{to}, con-
tractio est pluralis, pro |{byliś}my. In his vero: |{sie}-
|{dzie}m, |{miłuje}m |{się}; apocope est, pro |{siedzie}my, |{mi}-
|{łuje}my |{się}; Loco m antiquiores, ut hodiedum Bohe-
mi ponunt ch: |{słyszałe}ch, |{słyszeli}ch|{my}, |{słuchałe}ch|{ci},
et - |{ciem}, e. gr. |{świadom}ciem |{dobrze tego}; item: |{my}-
|{śmyć abo mycie}ch|{my}, |{abo mycieśmy to sprawili}. Ali-
quando seperatur a suo verbo, ut pro |{nie chciał}em
|{go bić}, dicitur |{bicie}m |{go nie chciał}. Vide differen-
tiam in his: |{zarazem szedł}; in tertia persona: statim
ivit; |{zaraz}-|{em szedł}, de prima: statim ivi; Orthogra-
phia diversa. Cn. Th. — {§}. 2. Litera m est characte-    
ristica Dativi Pluralis omnium generum, substantivo-
rum finientium in -om, e. gr. |{królom}, |{pannom}, |{sło}-
|{wom}; adjectivorum desinentium |{in} -ym vel -im, |{e}.
|{gr}. |{dobrym}, |{miękkim}. 

\endinput

Jedenasta litera Łacińskiego abecadło; Grecy ją zowią my,
w Hebrajskim języku mem. Kras. Zb. 2, 79. J. Ko-
chanowski od pospolitego m, 7:. p, dom, róźni drugie
kre'skowane, 12. p. uskrom'; na tym Gornicki kładzie
daszek. Now. Char.
§. l. M fini nominis additum', primam personam Ver-
bi jestem signiñcat: mocnym teraz - mocny jestem.
Trzeźwymcź, źywemcź vel żyw-emeil, id est, wżdy źy-
wy jestem. In his vero: bylim tam, slyszelim to, con-
›tractio est pluralis, pro byliśmy. In bis vero: sie-
dziem, mili/zjem się; apocope est, pro siedziemy, mi-
lujemy się. Loco m antiquiores, ut hodiedum Bohe-
mi ponunt ch: słyszałecb, słyszelichmy, słuchałecbeź,
et-cźem, e. gr. świadomćiem dobrze tego; item: my-
s'myć abo myez'ecbmy, abo myeies'my to sprawili. Ali-
quando seperatur a suo verbo, ut pro nie chciałem
go bić, dicitur biciem go nie chciał. Vide differen-
tiam in his: zarazem szedl; in tertia persona: statim
ivit; zaraz-em szedł, de prima: statim ivi; Orthogra-
phia diversa. Cn. Tb. - 2. Litera m est characte-
1000 !!!!!!!!!!!!!!!!!!!!!!!!!!!!!!!!!!!!!!!!!!!!     
ristica Dativi Pluralis omnium generum, substantivo-
rum ñnientium in -om, e. gr. królom, pannom, sło-
wom; adjectivorum desinen-tium in -ym vel -im, e.
gr. dobrym, miękkim. 

%%% Local Variables: 
%%% coding: utf-8
%%% mode: latex
%%% TeX-engine: xetex
%%% TeX-master: t
%%% TeX-PDF-mode: t
%%% End: