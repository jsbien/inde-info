% Tom I s. 99 XLIII
% Zofia Smuga 2015-05-25 2014L - Problemy dygitalizacji tekstów 3322-PDT-OG
% https://djvu.szukajwslownikach.uw.edu.pl/linde-t/01/index.djvu?djvuopts&page=99&zoom=page&showposition=0.49,0.49
skim i t. d. |{błazen} pochlebca, łacno zatem i nasze |{pobła-}
|{żać}, tu nakierować. \textemdash \ Pospolicie mniemany, że słowo |{szlach-}
|{cic}, |{szlachta}, poszło od Niemieckiego @{Schlacht}, bitwa; atoli
jednak u Kraińczyków i Windów |{shlahta}, % po "shlahta" jest albo przecienk, albo kropka
|{poshlahtnost} % albo poshlahtnosi
po-
krewieństwo, |{shlahtnik} krewny. Tak i w |{Otfrydzie} i |{Notkerze}
|{slahta}. % po "slahta" jest kropka, która miała chyba być przecinkiem
|{slahto}, |{gislaht} ród, rodzaj, familia, w niższej Sax. i
wyższych Niemcz. @{Schlacht}, @{Schlecht}, |{Svec.} |{slag}, |{slägt} % jest albo "slagt", albo "slagi" - nieczytelne
(cf. @{\so{Hoch}}⸗
@{\so{deutsch}, \so{Schlag}, von dem gutem \so{Schlage}, aus der Art \so{Schla}}⸗
@{\so{gen}, nach einem \so{Schlachten}}). |{Szlachta} tedy nie z Niemiec-
kiego @{\so{Schlacht}} bitwa; lecz z Niemieckiego @{\so{Geschlecht}}, wy- % nie wiem, czy dobrze odczytałam słowo po niemiecku

pada na Niemieckie @{der Geschlechter} patrycyusz. Podobnież w % nie wiem, czy dobrze odczytałam słowo po niemiecku

inszych dyalektach, w Bośnieńskirn, Kroatskim, Dalmatskim,
Szlachcic |{plemiennik}. \textemdash \ |{Gody} w Polskim Boże narodzenie,
biesiada, uczta, bonowanie, wesele małżeńskie; stąd oczy-
wiście |{godować}. |{Boh}. hodowati, biesiadować, także u nas
|{chodować}, żywić, karmić, podejmować swoim kosztem; ale
|{godny}, |{godzić się}, |{godzić} na co, |{ugodzić} w co, |{ugoda}, |{zgoda},
%"a" w "zgoda" jest nieczytelne 
|{pogoda}, |{przygoda}, |{wygoda}, |{godzina} i t. d. chociaż jawno i
podług słuchu i głosek, od |{gody} idą, ciężko od razu z nie-
go wykładać, nawet udawszy się do innych dyalektów. U
Czechów hod, hody, prócz |{uczty}, |{biesiady}, |{wszystkie wielkie}
|{Święta} znaczą; u Windów |{gody}, święto, uroczystość, imieni-
%kreska nad"Ś" w "Święta" jest widoczna jako zamazana kropka
ny, u Kraińczyków szczególnie imieniny, u Rossyan |{god}
rok, |{godina} czas, los, szczęście, |{godowat'} rok przebywać.
%po "godowat" jest albo kropka na górze, albo apostrof (oznaczenie zmiękczenia?)
Dopiéro zastanowiwszy się, że tu wszędzie znaczy się mniéj
lub więcéj jakiś czasu przeciąg, jakaś pora, doba, mająca
pewne ograniczenie, pewne przeznaczenie, pewną własność,
pokazuje się, iż kiedyś |{gody} znaczyło porę uprzywilejowaną,
do jakiejś uroczystości, porę do czegoś osobliwie zdatną, po
myśli czyjej, zgoła chwilę szczęśliwą i właściwą do szczęśli-
wego przedsiębrania; a już wynurza się, że pierwiastkowe
znaczenie słowa |{gody} zatraciło się, nie tylko u nas, lecz i u
naszych pobratymców; że zaś tkwi w pochodzących z niego,
wyrażających bądź czasu jakiś kres, n. p. |{godzina}, bądź te-
goż jaką zdatność, n. p. |{pogoda}, bądź też w ogólności wza-
jemne z sobą stosowanie się, z sobą się jednanie, n. p. zgo-
da, ugoda, wygoda, i t. d. \\
\begin{center}
% sprawdzić (JSB):
§\char"20D4 . 58. \\
\end{center}
\hspace {1 cm} Nad to pokazuje się czasem z dyalektów, że co z razu
zdaje się być słowem o dwóch znaczeniach, nie jest jedno,
ale dwojakie i z dwóch źrzódeł. Tak \so{naczelnik} u nas 1) prze-
łożony, na czele niby będący, 2) taśma naczelna. W Rossyj-
skiem zaś pierwsze brzmi i pisze się |{naczalnik}, a znaczy
nietylko przełożonego komenderującego, lecz autora jakiej
rzeczy, który jej dał początek; i pochodzi wraz ze słowami
|{Naczało} początek, %"początek" jest w słowniku zaisany jako "poczatek" ("a" zamiast "ą")
|{naczalnyj} początkowy, od zadawnionego
|{cząć}, |{czynać}, t. j. począć, poczynać, rozpocząć, rozpoczynać. \\
{\hspace {0,3 cm} \tiny |{Słownik Lindego wyd. 2.} Tom I.}
\endinput

%%% Local Variables: 
%%% coding: utf-8-unix
%%% TeX-engine: xetex 
%%% mode: latex
%%% TeX-master: "Linde1_etymologia_fragmenty"
%%% TeX-PDF-mode: t
%%% End: