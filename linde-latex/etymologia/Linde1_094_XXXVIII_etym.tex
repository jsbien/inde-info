% Tom I s. 94 XXXVIII
% Jacek Maksymowicz
% https://djvu.szukajwslownikach.uw.edu.pl/linde-t/01/index.djvu?djvuopts&page=94&zoom=width&showposition=0.5,0.18

z tego \textit{jest} już i reszta osób czasu teraźniejszego u nas po-
chodzi: \textit{jesteś}, \textit{jesteśmy}, \textit{jesteście}. U Windów zaś pierwsza
osoba liczby pojedynczéj czasu teraźniejszego \textit{sem}, w licz-
bie mnogiéj \textit{smo}, idą za łacińskiem \textit{sum}, \textit{sumus}. — Słowo
\textit{iść}, \textit{idę}, wychodzi na Łacińskie \textit{it}, \textit{is}, \textit{eo}, \textit{Eccl.} iti, \textit{Ross.}
itti, idu, \textit{Vind.} jiti; lecz czas dokonany, \textit{szedł}, \textit{szła}, \textit{szło},
\textit{szli}, \textit{szły}, inszego jest źrzódła, spólnego ze słowami: \textit{ślad},
\textit{szlak}, cf. \textsf{Niedersächs. sleke, Hochdeutsch Schlich, schleichen.} —
Słowo \textit{najduję}, w czasie przyszłym \textit{najdę}, \textit{najdzie}, okazuje
nam źrzódło \textit{iść}, \textit{idę}, \textit{idzie}, na wzór Łacińskiego \textit{invenio},
nachodzę, najść; wszakże czas przyszły \textit{nalazł}, od słowa
\textit{leźć}, \textit{łazić}, objęło miejsce zadawnionego \textit{nadszedł}, \textit{invenit}.
Rossyanin dotąd pisze i mówi: naiti, najdu, naszeł, nacho-
dit', nachożu.

§⃔. 47
Widzieliśmy dotąd, jak istotne głoski się zamieniają, jak
się też przestawiają; teraz największą uwagę nam obracać
potrzeba na nową etymologii trudność pochodzącą, już to
z ubywania prawdziwych głosek istotnych, już to z przy-
bywania pozorno istotnych, już z tego obojga razem.

§⃔. 48
Ubywania istotnych żaden dyalekt Słowiański tak oczy-
wistych nie daje dowodów, jak Sorabski w Luzacyi; tam
bowiem piszą i mówią: rib⸗grzyb, noy⸗gnoj; pzez⸗przez; pzi⸗
przy. — Na ten kształt i u nas \textit{śpital}, z Łacińskiego \textit{Hospi-
tale}, od słowa \textit{hospes; Jeronym, Jarosz}, zamiast \textit{Hierony-
mus}; \textit{Jacek} zamiast \textit{Hiacynth}. Mamy i słowa swojskie, któ-
rym naprzód dla wygodniejszego wymawiania odejmowano
głoski istotne przez skrócenie, co w ten czas było błędem,
lecz powoli wprowadziło się za prawidło, tak że już teraz
wciąż mówiemy: \textit{bał} się, zamiast dawnego \textit{bojał} się; \textit{stał},
zamiast \textit{stojał}; \textit{pas}, zamiast Słowiańskiego \textit{pojas}. Ale naj-
znakomitszym w tym względzie odmianom podpadło w Sło-
wiańskich dyalektach i obcych językach słowo \textit{córa, córka},
\textit{Grec. ϰόϱη}, \textit{Eccl.} дщєрь, \textit{Boh.} dcera, dcy, \textit{Ross.} docz,
% litera щ nie jest renderowana poprawnie, chyba, że znajdziemy czcionkę do cerkiewnosłowiańskiego
w drugim przypadku \textit{doczery}. \textit{Croat.} kcher, hcher, kchi,
\textit{Vind.} hzhi, hzher, \textit{Svec.} doter, \textit{Dan.} dotter, \textit{Anglosax.} doh-
tor, w \textit{Ottfrydzie} dohter, w \textit{Willeramie} tohter, \textit{Angl.} da-
ughter, \textsf{Nidersächs. Dochter}, \textit{Pers.} dochter, \textit{Germ.} \textsf{Tochter},
% "Nidersachs" może być literówką, która była w oryginale
\textit{Graec.} thygater. Zatem odmiany w istotnych głoskach tego
słowa zaszłe, tak sobie wyobrazić potrzeba.

{\catcode`\@=12

\begin{center}
\begin{tabular}{ccccc@{\hskip 1.5em}l@{\hskip 0.5em}l}
 & &c   &—&r& \textit{Pol.} &cora, córka.\\
 & &k   &—&ϱ& \textit{Gr.}  &\textit{ϰόϱη}.\\
d&—&szcz&—&r& \textit{Ecc.} &dszczer'.\\
d&—&c   &—&r& \textit{Boh.} &dcera.\\
d&—&c   &—&—& \textit{Boh.} &dcy.\\
d&—&cz  &—&—& \textit{Ross.}&docz.\\
d&—&cz  &—&r& \textit{Ross.}&doczery.\\
h&—&zh  &—&—& \textit{Vind.}&hzhi.\\
\end{tabular}
\end{center}

}
%%% Local Variables: 
%%% coding: utf-8-unix
%%% TeX-engine: xetex 
%%% mode: latex
%%% TeX-master: "Linde1_etymologia_fragmenty"
%%% TeX-PDF-mode: t
%%% End: