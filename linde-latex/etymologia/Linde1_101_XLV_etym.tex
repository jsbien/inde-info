% Tom I s. 101 XLV
% Krzysztof Nowak
% https://djvu.szukajwslownikach.uw.edu.pl/linde-t/01/index.djvu?djvuopts&page=101&zoom=page&showposition=0.49,0.49

2) Pójdźmyż teraz do trudniejszych, a przeto mniéj
oczywistych. \textit{I}mię \textit{Carn.} jime, \textit{Croat.} ime, \textit{Slav.} imene,
\textit{Ross.} imia, \textit{Boh.} gine, gmeno, (porównaj \textit{miano}, \textit{mianować},
\textit{mianowity}, \textit{Boh.} gmenowati) tak jak @{Ł}acińskie \textit{Nomen}, Grec-
kie \textit{onoma}, Niemieckie @{Name}, te same istotne głoski słowa
\textit{jąć}, \textit{imać}, \textit{imę}, zawiera; nad to, gdy i \textit{imię} jest w saméj
rzeczy coś przypadającego do tego co mianuje, i niby to
ima i przygarnywa do siebie, więc wyraża tę czynność,
którą i słowo \textit{jąć}, \textit{imać}; a tak do niego odniesione być
może \textit{Miéć}, \textit{miał}, \textit{mam}, \textit{Boh.} mjti, mel, mam, dla stra-
conego na początku \textit{j} trudność niejaką z pierwszego wej-
rzenia pokazuje; lecz ta, gdy się weźmie na uwagę \textit{Ross.}
\textit{imiet'}, \textit{imieł}, \textit{Kraińskie iimeiti}, wcale niknie; a wywód ten
większej jeszcze pewności nabywa od dawnego naszego
rzeczownika \textit{imienie}, \textit{Boh.} gmenj, \textit{Ross.} \textit{imienie}, \textit{Carn.} \textit{jime}-
\textit{njnje},  ⸗ possessya, dobro, co kto ma, mienie, majątek; po-
równaj poślednią połowę słów Łacińskich patri\textit{monium}, ma-
tri\textit{monium}, daléj Polskie \textit{majętność}, \textit{majętnostka}, \textit{majętnicz}-
\textit{ka.} --- \textit{Wymię}, \textit{wymiona} u krowy, \textit{Boh.} Weymie, weyme, we-
meno, \textit{Carn.} \textit{vime}, vimzhiz, \textit{Ross.} \textit{wymia}, \textit{wymeni}, bezpo-
śrzednie do słowa \textit{wyjąć}, \textit{wyjmować}, nie należą; lecz do
słowa \textit{wymiąć}, \textit{wyminąć}, \textit{wymięty}, \textit{Ross.} \textit{wymiat'}, \textit{wymnu},
\textit{Boh.} weymnauti; atoli samo \textit{miąć}, \textit{mnie}, \textit{mnę}, \textit{Boh.} mnau-
ti, mnu, \textit{Ross.} \textit{miat'}, \textit{mnu}, możnaby i nie od rzeczy odnieść
do źrzódła \textit{jąć}, \textit{imać.} --- Słowiańskie \textit{um} \textit{Germ.} @{Vernunft},
u \textit{Rossyan}, \textit{Kraińczyków} \textit{Windów} , i t. d. dotąd jeszcze uży-
wane, u nas zaś zaginione, jest bezpośrzedniczém źrzódłem
słów: \textit{Umiéć}, \textit{umiejętny}, \textit{umiejętność}, \textit{rozum}, \textit{rozumny}, \textit{ro}-
\textit{zumiéć}, \textit{dorozumiéć}, \textit{porozumiéć}, \textit{wyrozumiéć} \textit{rozumować},
\textit{sumienie}, \textit{sumienny}, \textit{sumienność}; rozważając zaś pokre-
wieństwo między \textit{ą} a \textit{u} zachodzące, (o czem wyżéj było,

% dodać paragraf
39-40.), słowo to \textit{um} ze swemi pochodzącemi, do pier-
wiastku \textit{jąć}, \textit{imać}, odprowadzamy; ile gdy i znaczenie jego
łatwo się ze źrzódłem \textit{jąć}, w sposobie przenośnego mó-
wienia godzi. \textit{Mniemać}, \textit{Boh.} mjniti, \textit{Ross.} \textit{mnit'}, \textit{pomnit'},
\textit{mniu}, \textit{pomniu}, \textit{Carn.} mieniti, porównaj \textit{Obs. Lat.} \textit{menere},
skąd \textit{memini}, \textit{obs.} \textit{meminisci}, skąd \textit{eomminisci}, \textit{Svec.} \textit{mena},
\textit{Gr.} \textit{menyo}, \textit{mnao}, skąd \textit{mnema}, \textit{Lat.}  \textit{memoria}, \textit{mentio},
\textit{mens}, \textit{Angl.} \textit{mind}, \textit{Svec.} \textit{mon}, \textit{Isl.} \textit{mune}, \textit{Germ.} @{Meynung},
@{meynen}; daléj składane nasze: \textit{pomniéć}, \textit{pominać}, \textit{wspo}-
\textit{mnieć}, \textit{wspominać}, \textit{wzmianka}, \textit{wzmiankować}, \textit{dopomniéć}, \textit{do}-
\textit{pominać}, \textit{napominać}, \textit{przepomniéć}, \textit{przepominać} \textit{przypo}-
\textit{mnieć}, \textit{przypominać}, \textit{upomniéć} \textit{upominać}, \textit{upominek}, \textit{upo}-
\textit{minkować},\textit{zapomniéć}, \textit{zapominać}, \textit{zapamiętać} \textit{zapamiętały}
\textit{zapamiętałość}, \textit{pamięć} \textit{pamiętny}, \textit{pamiętnik}, \textit{pamiątka}; wszys-
tkie te, w znaczeniach nadzmysłowych, przenośnych ze
zmysłowego znaczenia pierwiastku \textit{jąć}, \textit{imać}, mniéj więcéj
wyraźnie, istotne głoski źrzódlosłowa swego  zachowały;
zatém wywód, ich z niego, choć mniéj oczywisty, na wiarę
zasługuje. ---  Że \textit{przyjaciel} do źrzódła \textit{jąć}, należy, dowodzi
kościelny dyalekt; albowiem podług niego \textit{prijatel}, \textit{prijemnik} ⸗
przyjmiciel, przyjmujący. --- \textit{Rozmaity}, \textit{rozmaiość}, dla utra-
conego właściwie naczelnego \textit{j}, zamiast \textit{rozjmaity}, \textit{rozjmai}-
\textit{tość}, staje się, co do głosek, trudném do wywodu z tego
źrzódła; - podług znaczenia zaś bardzo dobrze wypada. ---
\textit{Zima}, \textit{zimno}, \textit{Boh.} zyma, zeyma, \textit{Ross.} \textit{zima}, \textit{Carn.} \textit{sima},
jeżeli je za skazówką istotnych głosek odniesiemy do źrzó-


%%% Local Variables: 
%%% coding: utf-8-unix
%%% TeX-engine: xetex 
%%% mode: latex
%%% TeX-master: "Linde1_etymologia_fragmenty"
%%% TeX-PDF-mode: t
%%% End: