% Tom I s. 96 XL
% Paulina Nowaczkiewicz
% https://djvu.szukajwslownikach.uw.edu.pl/linde-t/01/index.djvu?djvuopts&page=96&zoom=width&showposition=0.5,0.18

\textbf{Rozdział VII}.
|{Pomocy etymologiczne}. % nie jestem pewna czy w tym i poprzednim wierszu tekst nie powinien być rozstrzelony

§\char"20D4. 51.
Gdy słowa, nad których dochodzeniem pracuje etymolog,
nie są dzisiejsze, powinien pilnie uważać, czy za czasem
jakie w nich nie zaszły odmiany, bądź co do pisowni ich,
bądź co do znaczenia, bądź co do tego obojga razem. Do
przykładów wyżéj przełożonych, w paragrafach o ubywaniu
prawdziwie istotnych, a przybywaniu pozorno istotnych gło-
sek, przyłączam tu następujący: Słowo |{rzetelny}, tak pisane,
nie wiedzieć z kąd się wzięło; lecz jak go |{Jan Kochanowski}
|{Klonowicz} i t. d. piszą, |{źrzetelny}, |{Boh}. zrzetedlny, |{Ross}. |{zri-}
|{telnyj}, widoczna, że pochodzi od wyżej wspomnionego zriti,
|{źrzeć}, zkąd |{uźrzeć}, |{wzrok}, ztąd u dawnych znaczy: 1) oczy-
wisty, w oczy wpadający, jawny; 2) przenośnie, nieobłudny,
otwarty, szczéry, rzeczywisty. \textemdash\ |{Obces}, |{obcessowy}, u dawnych
|{obses}, |{obsessowy}; a że to Łacińskie |{obsessus}, dowodzi staro-
żytne znaczenie jego, bo opętany; tak pisze |{Kochowski:} «źli
z obsessów wyparci szatani.» |{Mączyński} zaś |{obsessus a dae-}
|{mone}, |{daemoniacus}, tłumaczy: opętany, obses. \textemdash\ Żyto w dy-
alektach znaczy wszystkie gatunki zboża; jak go jeszcze |{Sy-}
|{reniusz} używa, pisze albowiem: «pszenica jest żyto; orkisz
«jest żyto; jęczmień jest żyto.» To co teraz |{żyto}, wyrażało
u dawnych |{reż}; zkąd rżany chléb, |{zboże} zaś pewnie poźniej-
sze Słowiańskie słowo, od słowa |{Bóg}, w piątém przypadku
|{Boże} pochodzące, u Czechów znaczy |{towary}, u Sorabów
|{szczęście}. \textemdash\ |{Obmowa} dziś u nas potwarz, zły język, u na-
szych dawnych pisarzów zastępuje Greckołacińskie |{periphra-}
|{sis} «słowne ogradzanie, opisywanie, obmawianie, gdy jedna
«rzecz wielą słowy bywa ogradzana albo szérzej wyma-
wiana.» |{Mączyński}. W pośledniejszym czasie znaczyło na-
mowę, obradę, rozmowę; n. p. «Panowie przyjechali do J.
«K. Mości na |{obmowę}, a skoro było po |{obmowie}, (po sessyi)
«odjechali.» Potém: wymówkę, wymawianie siebie, exkuzę;
azaliż nie wcale przeciwną rzecz, jak dzisiaj! \textemdash\ Polskie:
|{hetman}, |{Ross}. getman, |{Cosac}. attaman, |{Boh}. Heytman, wy-
chodzi na Niemieckie @{Hauptmann}; wszakże tu pamiętać na to,
że @{Hauptmann} w dawnej Niemczyźnie, nie jak w dzisiejszej,
szczególnie kapitana, lecz naczelnika, na czele będącego,
przełożonego, co u Windów i innych Słowian |{poglavar}, zna-
czyło. \textemdash\ I Polskie |{gość}, |{Boh}. host, za jedno z Łacińskiem
|{hostis}, ten pewnie przyjmie, któremu z |{Cycerona de offic}.
|{1, 12, Varr. 4}, wiadomo, że w dawnej Łacinie |{hostis}, nie
znaczyło nieprzyjaciela, lecz obcego, |{peregrinum}, |{hospitem}.
§\char"20D4. 52.
Z tego wypada, że pierwszą do etymologii pomocą są
dzieła, które nam stawią historyą odmian słów, tak w
kształcie, jak w znaczeniach, a zatem potrzeba |{glossaryów},

%%% Local Variables: 
%%% coding: utf-8-unix
%%% TeX-engine: xetex 
%%% mode: latex
%%% TeX-master: "Linde1_etymologia_fragmenty"
%%% TeX-PDF-mode: t
%%% End: