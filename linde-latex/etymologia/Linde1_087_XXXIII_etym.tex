% Tom I s. 87 XXXIII
% Magdalena Hanik
% https://djvu.szukajwslownikach.uw.edu.pl/linde-t/01/index.djvu?djvuopts&page=87&zoom=width&showposition=0.5,0.33

% Trudny tekst.  Wyjątkowy skomplikowany podział na łamy!

% Do uzupełnienia greka, cyrylica, SCS


@{Magdeburgisch Halle}, @{Reichenhall}, @{Hallstadt}, solne kopalnie wytykają;
\textit{Tatar.} \textit{Krim.} salz, \textit{Svec.} \textit{et Island.} salt, \textit{Angl.} salt, @{Niederdeutsch solt};

[do uzupełnienia - JSB]

\textit{Ross.} [corn JSB], \textit{Carn.} sol, \textit{Vind.} sol, sou.
\textit{Graec.} ήλιος (⸗słońce, @{Sonne}),\textit{Graec.} [Sitz JSB] (blask  @{Glanz}),

[do uzupełnienia - JSB]


halal (⸗świecił lśnił się, @{leuchten}, @{glänzen}).

Z tych tedy słów zdaje się, jakoby zwykle te dwie gło-
ski \textit{h}-\textit{l} służyły do wyrażenia jasności; a ponieważ \textit{h}-\textit{l}⸗\textit{s}-
\textit{l}:\textit{s}-\textit{ln}⸗\textit{s}-\textit{n} zastanowmyż się nad następującym szeregiem:

\textit{Graec.} \textit{ήλιος}, \textit{Vallis.} haul, \textit{Litt.} saule, \textit{Dan.} soel, \textit{Svec.} sol, \textit{Lat.} sol, solis, \textit{Ross.} солнце, \textit{Pol.} słońce, \textit{Boh.} slunce,
\textit{Vind.} sunze, sonze, \textit{Carn.} sonze, \textit{Croat.} szuncze, \textit{Crim.} \textit{Tatar.} sune, son, \textit{Angl.} sun, \textit{Germ.} @{Sonne}.

6) Równie i słowa Greckie zaczynające się  \textit{a spiritu leni}
przybrały w inszych językach  \textit{s⸗h} n. p.

 \textit{Graec}. [!!! - JSB] \textit{Ital.} acciare, \textit{Germ.} @{hacken}, @{hacke}; \textit{Angl.} hack, \textit{Svec.} hacka, \textit{Lat.} seco, secare, securis,
\textit{Boh.} sekati, sekawati, seknauti, \textit{Carn.} sękati, sęzhi, \textit{Dalm.} sichi, sikao, \textit{Croat.} szuchem, \textit{Ross.} [!!! - JSB] \textit{Pol.}
siekę, sieczesz, siec.

7`) Dotąd się mówiło o \textit{s}, że częstokroć zastępuje gło-
skę \textit{h}; teraz przydaję, że się kładzie i za \textit{k⸗c⸗h}, n. p.

\textit{Pol}. Serce, \textit{Boh.} srdce, \textit{Ross.} сердце, \textit{Vind.} serze, \textit{Croat.} szercze, \textit{Germ.} @{Herz}, \textit{Svec.} hierte, \textit{Angl.} heart, \textit{Graec.}
καρδιά [??? - JSB] \textit{Gall.} coeur, \textit{Lat.} cor, cordis.

§. 30.
T jest zgłoska jedna z najtwardszych w abecadle Polskiém,
tak że po niej \textit{i} żadną miarą następować nie może; gdzie
tedy w inszych dyalektach Słowiańskich, lub też w inszych
językach miękkie \textit{t} przed \textit{i} się zdarza, tam w Polszczyźnie
pisze się \textit{ć}, ztąd wszystkie Polskie słowa od \textit{ć} w słownikach
Słowiańskich pod \textit{t} się znajdują; a tryb bezokoliczny, który
się u Polaków na \textit{ć} kończy, w inszych dyalektach na \textit{t} lub \textit{ti}.
2) Chociaż o pokrewieństwie głoski \textit{t} z głoską \textit{ć}, już
pod \textit{ć} wzmiankowało się, atoli tu jeszcze wyłuszczmy słowo:

Ojciec \textit{in Genit. per contract.} ojca; \textit{Boh.} otec, otce, \textit{Ross.} [отца JSB], \textit{Carn.} ozha, \textit{Croat.} o\textit{t}ecz, o\textit{t}acz, o\textit{ch}e, \textit{Vind.} oz\textit{h}a,
a\textit{t}ei, \textit{Sorab.} ey\textit{d}a, \textit{Lapp.} a\textit{t}zhie, a\textit{t}ye, \textit{Hung.} a\textit{t}ya, \textit{Turc.} a\textit{t}a, \textit{Roman.} a\textit{tt}a, \textit{Graec.} [JSB], Goth. \textit{atta}; \textit{cum praefixo} \textit{t} Turc.
\textit{t}ada, \textit{Hispan.} \textit{t}ai\textit{t}a, \textit{Angl}. \textit{d}a\textit{d}, \textit{d}a\textit{dde}, @{Niedersächs. teite}, \textit{Pol.} \textit{t}a\textit{t}a, \textit{Graec.} [??? - JSB], \textit{Lat.} \textit{t}a\textit{t}a. (cf. \textit{Pol.} \textit{d}zia\textit{d}, \textit{d}zie\textit{d}zic, \textit{d}ziecię,
\textit{d}zia\textit{t}wa, \textit{D}z⸗\textit{D}).

§. 31.
1)\textit{W⸗B; qu. v.}
2). W częstokroć na początku słów Polskich (co w Nie-
mieckim bywa) przybyszowe, niby to jak \textit{spiritus asper},
gdyż w inszych dyalektaeh i językach nie znajduje się.

% niektóre t powinny być kursywą
\textit{Pol.} \textit{W}ęgiel, \textit{Germ.} @{Wingel}, \textit{Vind.} \textit{v}ogel, \textit{Carn.} ogel, ogal, \textit{Croat.} \textit{v}ugel, \textit{Hung.} szugoly, \textit{Boh.} uhel, \textit{Ross.} [!!! - JSB],
\textit{Lat.} angulus, cf. \textit{Pol.} \textit{W}ązki, \textit{Carn.} \textit{v}osek, \textit{Croat.} \textit{v}uzek, \textit{Boh.} auzky, \textit{Ross.}  [!!! - JSB], \textit{ap. Keron.} enke, enga, \textit{ap. Otfrid.}
ango, \textit{Germ.} @{enge}, \textit{Vallis.} ing, \textit{Bretan.} anc, \textit{Let.} ank, \textit{Goth.} aggon, \textit{Lat.} angustus, ango, anxi, actus, cf. \textit{Graec.} [??? - JSB],
 \textit{Pol.} Wnętrze., \textit{Ross.},[ Hyrps - JSB], cf. \textit{Gall.} interieur, \textit{Lat.} interius, intus, cf. \textit{Pol.} nadro, Eccl. [„mpc --- JSB]
\textit{Polon.} \textit{W}ęgry, \textit{W}ęgrzyn, \textit{Boh.} Uhry, \textit{Germ.} @{Ungarn, Hungarn}, \textit{Boss.} [!!! - JSB], \textit{Vind.} Vogel-, \textit{Croat.} Vugrin.


3) \textit{W} często zajmuje miejsce samogłosek, osobliwie \textit{u,
y, i, o};

\textit{Lat.} S\textit{u}s, s\textit{u}is, Graec. [!!! - JSB], \textit{Vind.} s\textit{v}ine, \textit{Carn}. s\textit{v}ina, \textit{Croat.} sz\textit{v}inya, \textit{Boh.} swine, \textit{Ross.} свинья, \textit{Eccl.} [!!! - JSB]
\textit{Pol.} świnia, \textit{Svec}. s\textit{w}in, \textit{Angl.} s\textit{w}ine, \textit{Germ.} @{Schwein} — \textit{Lat.} s\textit{u}us, s\textit{u}a, s\textit{u}um, \textit{Graec.} [!!! - JSB], \textit{Vind.} s\textit{v}oi,

\endinput

Odp: Słownik Lindego s. XXXIII
Napisane przez: Magdalena Hanik ( poniedziałek, 11 maja 2015, 20:23 )
 

Miałam uzasadnić swój wybór znaku [:] w opracowanej przeze mnie części tekstu.

Na podstawie treści wywnioskowałam, że znak ten mógłby sygnalizować oboczność, która oznaczana jest właśnie poprzez [:].

Wybrane fragmenty potwierdzające moją tezę:

 

6) Równie i słowa greckie zaczynające się a spiritu leni przybrały w inszych językach  s:h, n. p.

Germ. hacken, hacke;

Angl. hack

Svec. hacka

Lat. seco, secare, securis

Pol. siekę, sieczesz, siec

(…)

 

7) Dotąd się mówiło o s, że częstokroć zastępuje głoskę h (…)

%%% Local Variables: 
%%% coding: utf-8-unix
%%% TeX-engine: xetex 
%%% mode: latex
%%% TeX-master: "Linde1_etymologia_fragmenty"
%%% TeX-PDF-mode: t
%%% End: