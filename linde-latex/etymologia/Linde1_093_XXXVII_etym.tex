% Tom I s. 93 XXXVII
% Mateusz Łubiński
% https://djvu.szukajwslownikach.uw.edu.pl/linde-t/01/index.djvu?djvuopts&page=93&zoom=page&showposition=0.49,0.49 

różnią się od Litewskiego waldyti, \textit{Finl.} wallitsema, \textit{Svec.}
walda, w \textit{Otffrydzie} uualtan, w \textit{Izydorze} uualden, \textit{Germ.}
@{walten}, zkąd @{Gewalt}, po naszemu \textit{władza}, @{Gewaltsamkeit} po
naszemu \textit{gwałt} — Słowo \textit{pokrzywa}, \textit{Boh.} koprziwa; \textit{Ross.}
кропива, \textit{Vind.} kopriwa, kropiva; \textit{Croat.} kropiva, porównaj
z imieniem opactwa \textit{Koprzywnickiego}, czyli \textit{Pokrzywnickiego}.

§. 44.

Poznanie słów obcych, dlatego też często przytrudniej-
szém bywa, że nie tylko głoski, lecz i całe zgłoski, na
swojski krój przerobiono; oraz tymże swojskie zakończenia
i składania nadano; tak naprzykład, i \textit{ukrzyżowany}, i \textit{cru-
cifixus}, i \textit{crucifié}, i gekreuzigt, pochodzą od słowa \textit{crux},
\textit{crucis}, \textit{Pol.} krzyż, \textit{Boh}, krzjż, \textit{Vind.} krish, \textit{Croat.} Kris,
\textit{Hung.} kereszt, \textit{Ross.} крестѣ, porównaj \textit{Polon.} chrzest, chrzcić,
i t. d. — \textit{Pol.} Rzym, L\textit{at.} Roma, \textit{Germ.} Rom, \textit{Pol.} Rzymia-
%                             ^ Litera L w oryginale nie jest pisana kursywą
nin, \textit{Lat.} Romanus, @{der} @{Römer}, \textit{Pol.} Rzymski, \textit{Germ.} @{Rö⸗
misch}. — \textit{Pol.} \textit{Uryanka}, \textit{uryantska} perła, zamiast \textit{Oryentka},
\textit{oryentalna}. — \textit{Gall.} coridore, \textit{Pol.} Kurytarz. — \textit{Germ.} @{Kreuz⸗
gang}, \textit{Pol.} Kruzganek.

§. 45.

Przyswajając obce słowa, pozwolono sobie odmiany, nie
tylko co do głosek i zgłosek, ale też i co do znaczeń po-
dług potrzeby. Rzeczownik (\textit{Substantivum}) \textit{wola}, oczywiście
skazuje Łacińskie \textit{Volo}, \textit{voluntas}, Greckie \textit{bule}, \textit{bulomai}; i
nasze \textit{wola} i Łacińskie \textit{voluntas} są też jednego znaczenia;
a zatém dziwna, jak czasownik \textit{wolę}, \textit{wolić}, nie został się
przy znaczeniu Łacińskiego czasownika \textit{volo}, \textit{velle}; lecz ra-
czéj poszedł na Łacińskie \textit{malo}, \textit{malle}, t. j. \textit{magis volo}; co
nie przez co innego stać się mogło, tylko że u nas już
pierwej Łacińskie \textit{volo}, \textit{velle}, przez słowo \textit{chcieć}, \textit{chcę},
zastąpione było; myśmy tedy potrzebowali szczególnie skła-
dniejszego rzeczownika, niż było owo \textit{chcenie}, i takim jest,
\textit{wola}; prócz tego, trzeba było jeszcze czasownika na wyra-
żenie wyboru jednéj rzeczy z dwóch, co u nas \textit{wolić}, wła-
śnie wyraża. U Kraińczyków i Windów \textit{wolim} znaczy \textit{chcę},
\textit{wybieram}, @{will}, @{wähle}; u Rossyan \textit{woliu} znaczy \textit{chcę}, \textit{ży-
czę}.— Kto zechce, niech sie zastanawia, bo rzecz tego
warta, jakie w używaniu naszém mają znaczenia słowa Ła-
cińskie: \textit{subjekcya}, \textit{konsolacya}, \textit{kollacya}, \textit{palestra}, \textit{mecenas;}
jak różne od rodowitego !
% w oryginalnym tekście także jest spacja między "rodowitego" a wykrzyknikiem.

§. 46.

Są też czasowniki takie, które nie z jednego źrzódła
konjugacyjne czyli czasowe swoje odmiany czerpają. Tak
słowo \textit{być} z trojakich pierwiastków czasy swoje bierze: 1)
\textit{być}, \textit{będę}, \textit{bądź}, \textit{Vind.} biti, bodem, bom, bodi, cf. \textit{Ger.}
@{bin}, @{bist}, w \textit{Kerze} bim, \textit{Anglosax.} beon, \textit{Angl.} bee. 2) w trze-
ciéj osobie liczby mnogiéj czasu teraźniejszego \textit{są}, \textit{Vind}. so,
\textit{Ross.} sut' cf. \textit{Lat.} sunt, sint, \textit{Germ.} @{sind}, @{seyn}, \textit{Vallis.} sy. 3)
w trzeciéj osobie liczby pojedynczéj czasu teraźniejszego \textit{jest},
\textit{Vin.} je, \textit{Ross.} jest', \textit{Graec.} esti, w \textit{Ulfilasie} is, \textit{Germ.} @{ist}, a


%%% Local Variables: 
%%% coding: utf-8-unix
%%% TeX-engine: xetex 
%%% mode: latex
%%% TeX-master: "Linde1_etymologia_fragmenty"
%%% TeX-PDF-mode: t
%%% End: