% Tom I s. 95 XXXIX
% XXXIX Mariola Murawska
% https://djvu.szukajwslownikach.uw.edu.pl/linde-t/01/index.djvu?djvuopts&page=95&zoom=page&showposition=0.49,0.49

h  —-  zh  —- r  Vind. hzher.
h  —-  ch  —- r  Croat. hcher.
k  —-  ch  — r  Croat. kcher.
k  —-  ch  — —  Croat. kchi.
d  —-  t  —- r  Svec. doter
d  —-  tt  —- r  Dan. @{Dotter}.
d  —-  ht  —- r  Anglsx. dohtor.
t  —-  ht  —- r  Willer. tohter.
d  —-  ght  —-   r  Angl. daughter.
d  —-  cht  —- r  Pers. dochter.
t  —-  ch  —- r  Germ. @{Tochter}.
 th  —- gt  —- r  Graec. thygater.



§. 49.
Przybywanie pozorno istotnych równie mąci etymologią,
jak ubywanie prawdziwych. O przybywającém z początku
słów, \textit{g}, \textit{j}, \textit{dz}, wyżej już pod temiż głoskami mówiono, i

% niektóre J proste
na przykłady przytoczono słowa \textit{Jadam}, \textit{Jewa}, \textit{Jędrzej},\textit{ jabłko},
\textit{gniazdo}, \textit{dzwon}. Przydam tu jeszcze \textit{zwierciadło}; które to
słowo w Polskiém wcale się nie może etymologicznie wy-
tłumaczyć; łacno zaś w inszych dyalektach; albowiem {Boh}.
zrcadlo, \textit{Ross}. zerkało, zercało, \textit{Vind}. serzalu, \textit{Cro}. \textit{zerczalo}
należą do jednego źrzódła z naszemi slowami,\textit{ źrzenica}, \textit{wzro}k,
\textit{wejźrzeć}, \textit{wyźrzeć}, \textit{zajźrzeć}, \textit{zorza}, to jest, do zadawnionego
zriti źrzeć ⸗ widzieć.


% Nie wiem, czy skopiować także fragment tekstu z paragrafu 49 w języku niemieckim, który zachodzi na stronę polską
§. 50.
W niektórych słowach wszystko razem, tak ubywanie
prawdziwie istotnych, jak przybywanie pozorno istotnych glo-
sek, jak też jeszcze do tego i przemiana ich, czyni etymo-
logią wielce, trudną. Przy czém zastanowmy się, jak z \textit{Ale-
ksandra} mógł się stać \textit{Oleś}, z \textit{Elizabetha}, po naszemu \textit{Elźbieta},
zdrobniale \textit{Elżbietka}, \textit{Halśka}. -- \textit{Joannes} po naszemu \textit{Jan},
zdrobniale \textit{Janek}, \textit{Januś},\textit{ Janusiek}, \textit{Jaś},\textit{Jasiek},\textit{ Jasieczek}, \textit{Jachne-
czek}, \textit{Jasinek}, \textit{Jasineczek}. Toć zdziwiło \textit{Kochowskiego}, tak że w
fraszkach swoich pisze: «Patrzaj jak wiele imion masz z je-
dnego \textit{Jana}: \textit{Janusza} i \textit{Hanusa}, \textit{Iwana}, \textit{Isztwana}, \textit{Jonka}, \textit{Jaś-
ka}, \textit{Jasinka}, \textit{Jacha} i \textit{Jasiątko}; Jeden ród wołek, ciołek, kró-
wka i cielątko.» Także od \textit{Petrus}, \textit{Piotr}, w piątym przypad-
ku \textit{Pietrze}, zdrobniale \textit{Piotrek}, \textit{Pietrek}, \textit{Piotruś}, \textit{Piechnik},
\textit{Piechniczek}, \textit{Pieś},\textit{ Piesinek}, \textit{Piesineczek}. W imionach wła-
snych, byle się trzymać ich oznaczenia, snadno trafić, jak
po nitce do kłębka, do ich pierwiastkowego kształtu; lecz
to daleko trudniej w pospolitych słowach. Polskie\textit{ wątpić},
\textit{wątpliwy}, \textit{wątpliwość}, w żadnym dyalekcie się nie znajdują,
i jak się u nas piszą, nie mogą być wcale wywiedzione ety-
mologicznie. Trafiłem przecież na jeden ślad, podobno i nie
lekki, a to porównywając Polskie \textit{wątpliwość} z Kraińskiém
\textit{dwomliwost}. U Kraińczyków albowiem tak jak u Kroatów,
Dalmatów i t. d. \textit{dwojiti} znaczy wątpić, z czém porównaj
Polskie \textit{na dwoje}, na przykład: «na dwoje babka wróży; »
jeszcze to na dwoje. Porównaj z tém Łacińskie, \textit{dubius}, \textit{duo},
Niemieckie @{zweyfeln}, @{zmey}.


%%% Local Variables: 
%%% coding: utf-8-unix
%%% TeX-engine: xetex 
%%% mode: latex
%%% TeX-master: "Linde1_etymologia_fragmenty"
%%% TeX-PDF-mode: t
%%% End: