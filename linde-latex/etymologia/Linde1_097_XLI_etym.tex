% Tom I s. 97 XLI
% Krzysztof Nowak
% https://djvu.szukajwslownikach.uw.edu.pl/linde-t/01/index.djvu?djvuopts&page=97&zoom=300&showposition=0.21,0.17

% brak przypisu!

mianowicie zaś śrzedniej i ostatniej Greczyzny i Łaciny. Gdy
tylokrotnie pod pojedynczemi spółgłoskami Greczyznę i Ła-
cinę śrzedniego wieku przytoczyłem; już nie wątpię, że się
sam z siebie czytelnik zastanowi nad dzisiejszém u nas zna-
czeniem słów: \textit{patron}, \textit{prebenda}, \textit{proboszcz}, \textit{inkaust}, \textit{bulla},
\textit{papież}, \textit{arkusz} i t. p. 


\begin{center}
§\char"20D4 . 53. \\
\end{center} 


Ponieważ nie zawsze z właściwego źrzódła czerpano sło-
wa, lecz przyswajano, jakie się tylko w potrzebie nawinęły;
etymolog więc obracać ma uwagę, nie tylko na język głó-
wny, ale i na idyotyzmy, nie na samę tylko n. p. Niem-
czyznę ogólną, lecz i na dział onejże na niższą i wyższą,
nawet i na osobne prowincyonalizmy. Bierzmy Polskie \textit{chrzan},
\textit{Ross.} \textit{chrien,} \textit{Vind.} \textit{kran,} \textit{hren,} wszakże wychodzi na Austry-
ackie @{Kren}, @{Krän}; z Saskiem zaś @{Meerrettig} nie ma i najmniej-
szego podobieństwa. Bierzmy jeszcze Polskie \textit{fartuch}, \textit{Boh.}
fertuch, \textit{Austr.} @{Fürtuch}, i to różne od Saskiego @{schurse} Toż
podobnie Polskie \textit{stodoła} schodzi się z Austryackiem @{stabel},
ale nie z Saskiem @{Scheune}. \textit{Austr.} @{Rain}, rynka, \textit{Sax.} @{der Siegel.}
Wielka część słów naszych żeglarskich, kupieckich, należy
do niższej Niemczyzny, nawet do Holenderszczyzny, n. p.
szkuta, reja, zegielgarn, bezmian i t. d. 


\begin{center}
§\char"20D4 . 54. \\
\end{center} 

Największej zaś pomocy etymolog Polski spodziewać się
powinien, z pobratymczych dyalektów; ani też niechaj się
bez nich nie porywa wykładać słów początek. Któżby nie
sądził z pierwszego wejźrzenia, że słowo \textit{szkło}, należy do
spólnego źrzódła z Niemieckiem @{Glas}, a zatém \textit{szklnić} się,
nie \textit{lśnić} się, z tego \textit{szkło}; lecz Rossyjskie \textit{ste}k\textit{lo}, \textit{Eccl.} \textit{st'klo},
odkrywa nam jego daleko pewniejsze źrzódło, w słowie \textit{ste}-

% * to przypis!
\textit{kat'} ściekać, ściec, ciec, (\textit{Ross.} \textit{steklarus} szmelcarz *), podając
nam wyobrażenie cieklizny, płynu, czyli czegoś po roztopie-
niu stwardzonego. Za czém \textit{szklnić}, \textit{szlnić}, \textit{ślnić} powinnoby
się pisać \textit{lśnić}, i ma źrzódłowe głoski \textit{l-s}, o których po-
wie się niżej. --- A i to drugie Polskie \textit{korowody}, możnali
dokładniej wyprowadzać, jak z Rossyjskiego \textit{Korowod}, \textit{choro}-
\textit{wod}, wiedzenie choru, t. j. tańca, prowadzenie reju, rej.---
Równie i to trzecie, \textit{mlokos}, które jest tożsamo, co u Ros-
syan \textit{molokosos}, od słowa Rossyjskiego \textit{moloko}, po naszemu
\textit{mléko} a \textit{ssać}. O słowie zaś \textit{wązki}, wyżej mówiłem, że go,
oglądając się na Czeskie auski, \textit{Ross.} \textit{uzkij}, do Łacińskiego
\textit{anctus}, \textit{anxius}, \textit{angustus}, \textit{Germ.} \textit{engc}, \textit{Graec.} \textit{ancho}, odnieść
należy. 


\begin{center}
§\char"20D4 . 55. \\
\end{center}

Przypada także nieraz potrzeba, gdy pierwiastek słowa
u nas zaginął, szukać go po innych dyalektach; n. p. sło- 


%%% Local Variables: 
%%% coding: utf-8-unix
%%% TeX-engine: xetex 
%%% mode: latex
%%% TeX-master: "Linde1_etymologia_fragmenty"
%%% TeX-PDF-mode: t
%%% End: