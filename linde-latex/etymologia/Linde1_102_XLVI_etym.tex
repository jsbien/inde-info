% Tom I s. 102 XLVI
% Krzysztof Nowak
% https://djvu.szukajwslownikach.uw.edu.pl/linde-t/01/index.djvu?djvuopts&page=102&zoom=page&showposition=0.49,0.49

dłowego \textit{jąć}, \textit{imać}; trafiemy na wyobrażenie ściągania, \textit{zej}-
\textit{mowania}, ściskania, ścinania, co téż właśnie przymiotem
\textit{zimy} i \textit{zimna}; uważając zaś, że jak wyżej pokazano \textit{z « h} § 22.
i 32., można jeszcze do tych przyłączyć \textit{Lat. hyems, hye}-
\textit{mis, Gr. cheimon, cheima}. Podobnie i słowo \textit{ziemia, zie}-
\textit{mny. Boh}. zemie, \textit{Ross}. \textit{zemlia}, \textit{Carn}. \textit{semla}, za szlakiem
radykalnych, które zatem da nam wyobrażenie \textit{compagis},
ciała zjętego, zlepionego, spojonego; a znowu z pokre-
wieństwa \textit{z} i \textit{h}, Łacińskie \textit{humus, humi}, także przyrówny-
wać się może. --- Nie śmiałbym przy tych wspomnieć słowa
\textit{szcząt, szczęt, szcząd}, z pochodzącemi \textit{oszcządek, oszczątek},
\textit{oszczędny. oszczędzić}; gdyby mnie dawna pisownia, która
\textit{oszczymny, oszczymiać} ma, nie ośmielała.


Not. Gdybyśmy z tylu tu przytoczonych, od tego źrzó-
dla \textit{jąć, imać}, pochodzących, nawet tylko same oczywiste
i pewne przyjęli; czyż i tak jeszcze nie powinnibyśmy go za
jedno z najobfitszych, a razem gdy tyle słów Greckich i
Niemieckich do niego się odnosi, za jedno z najdawniej-
szych, a pewnie przedsłowiańskich poczytać? 


\begin{center}
§\char"20D4 . 61. \\
\end{center}

Szczery etymolog nie przedaje podobieństwa za pewność;
lecz trzyma się przy tém, co mu ukazują nieodmienne
prawidła, z natury istotnych głosek, i stosunku ich do
znaczenia wyczerpane: a mniéj jawne wywody szczególnie
podaje do zastanowienia się, póki szcześliwe jakie odkrycie
w dawnej pisowni, w historyi znaczeń, w pobratymczych
dyalektach, albo jego domniemania nie potwierdzi, albo lep-
szego nie nastręczy. 


\begin{center}
@{Rozdzial IX.} 
\textit{0 układzie Słownika radykalnego.} 
\end{center}

\begin{center}
§\char"20D4 . 62. \\
\end{center} 


Poty próżna nadzieja wielkich w języku naszym przez
etymologią odkryciów, póki wprzód nie będziemy mieli jak
najzupełniejszego zbioru slównego, wyciągnionego z jak
najrozmaitszych i najliczniejszych, pism ojczystych dawnych
i świeższych. Z takowego to zbioru dopiero przyjdzie sło-
wo po słowie, z nieustanną bacznością na pokrewne dya-
lekty, rozważać etymologicznie; toż naprzód pierwiastkowe
z nich a pochodzące pod niemi wypisywać. Z czego zro-
biłby się słownik etymologiczny, czyli źrzódłosłowowy; lecz
jeszcze i na nim nie byłoby dosyć. Albowiem trzebaby
nadto rozbierać każde pierwiastkowe słowo, i w niém isto-
tne czyli radykalne głoski oznaczać; zatém wszystkie słowa
pierwiastkowe według znalezionych w nich głosek radykal-
nych (bądź tych samych, badź pokrewnych) uszykować, na-
koniec w tak uszykowanych, stosunku znaczenia do rady-
kalnych głosek dochodzić, a już nie zatrudniając się uwa-
żaniem przypadkowych dokładek, lecz uważając samę tylko
treść czyli jądro słów, istotne głoski Słowiańskie, z istotnemi 



%%% Local Variables: 
%%% coding: utf-8-unix
%%% TeX-engine: xetex 
%%% mode: latex
%%% TeX-master: "Linde1_etymologia_fragmenty"
%%% TeX-PDF-mode: t
%%% End:
