% xelatex --shell-escape JSB14_object.tex
\documentclass[12]{mwart}
\usepackage{polyglossia}
\setdefaultlanguage{polish}

\usepackage{enumitem}
\usepackage{draftwatermark}

\usepackage{xltxtra}

\setmainfont[Mapping=tex-text]{TeX Gyre Termes}
\setsansfont[Mapping=tex-text]{TeX Gyre Adventor}
%\setmainfont{TeXGyreTermes}
%\setmainfont{DejaVu Serif}
%\setmainfont{Bitstream Vera Serif}
%\setmonofont{TeX Gyre Cursor}
\setmonofont{DejaVu Sans Mono}

\usepackage{ulem}


%\usepackage{pdfdraftcopy}
%\draftstring{http://bc.klf.uw.edu.pl}

\usepackage{xcolor}

\def\p#1{\textsf{#1}}
\def\f#1{\colorbox{black!15}{\textsf{#1}}}

\usepackage[style=authoryear,natbib=true]{biblatex}
%\addbibresource{JSB2013.bib,typografia.bib}
\addbibresource{4JSB2014u.bib}
\AtEveryBibitem{\clearfield{note}}
% Wyed !!!

\usepackage{graphicx}

\usepackage{hyperref}

\usepackage{soul}

\newcommand{\program}[1]{\textsf{#1}}

\title{Wstępna koncepcja\\
  wielopoziomowej dygitalizacji obiektowej\\(object multilayer
  digitization)\\na przykładzie słownika Lindego}

\author{Janusz S. Bień}

\date{29 stycznia 2015\\(uaktualnienie adresów 26.05.2018)}
\begin{document}
\maketitle
% \pagestyle{empty}

% no math
\catcode`\&=12
\catcode`\_=12

\begin{quote}
  Tekst na otwartej licencji Creative Commons Uznanie Autorstwa,
  źródła dostępne w repozytorium
  \url{https://bitbucket.org/jsbien/linde-info}.

%  Wersja:  \input{"| hg log -v -l 1 \jobname.tex --template '{node}'"}
\end{quote}

\section{Wstęp}

Punktem wyjścia do dygitalizacji obiektowej jest szczegółowa
\textbf{analiza typograficzna słownika}.

\begin{quote}
  Jak można zauważyć np. na egzemplarzu z Biblioteki Łańcuckiej
  udostępnianym przez Google Books, w oryginale wszystkie tomy mają
  przedtytuł (\textit{Schmutztitel},
  \url{http://de.wikipedia.org/wiki/Schmutztitel}; \textit{bastard
    title}, \url{http://en.wikipedia.org/wiki/Half_title}) treści
  \textsc{słownik Lindego}. Jest on pomijany w przedrukach, co zaburza
  paginację --- zakładamy uzupełnienie skanów o tę brakującą kartę.
\end{quote}

\begin{quote}
  W przedruku GUTENBERG-PRINT tom VI został opublikowany jako dwa
  woluminy z dorobionymi pseudo-oryginalnymi tytulariami. Zgodnie z
  oryginałem tom ten należy traktowac jako całość i uzupełnić o
  oryginalne karty tytułowe.
\end{quote}



Tradycyjne wyniki dygitalizacji mają postać tekstu uzupełnionego o
odpowiednie adnotacje (HTML, TEI itp.). Wynikiem dygitalizacji
obiektowej jest zasób elektroniczny w formie bazy danych opisujący
własności wyróżnionych w tekście obiektów. Zasób taki może mieć
różnorodne zastosowanie, niekoniecznie przewidziane przez jego
twórców; oczywistym i podstawowym zastosowaniem jest automatyczne
wygenerowanie dygitalizacji tekstowej w zadanym formacie. Jest
wskazane nadanie zasobowi International Standard Language Resource
Number (\url{http://www.islrn.org/faq/}, usługa jest bezpłatna).

Zakładamy wykorzystanie relacyjnej bazy danych --- wykorzystamy
doświadczenia zdobyte przy bazodanowej dygitalizacji indeksu \textit{a
  tergo} \citep{bc379}. 

Omawiane dalej poziomy to pewne wygodne sposoby prezentacji jej
zawartości, które mogą ulegać zmianom. Na granicy poziomów występują
często wieloznaczności, co traktujemy jako rzecz normalną, wymagajacą
traktowania w analogiczny sposób, jak wieloznaczności morfologiczne i
składniowe znane od lat lingwistyce informatycznej.

Zakładamy intensywne wykorzystanie systemu zarządzania wersjami
(mercurial) oraz możliwości logowania transakcji w bazie danych do
pełnego dokumentowania procesu tworzenia i uaktualniania zawartości
bazy.

Na różnych etapach dygitalizacji niezbędna będzie ręczna korekta
danych, ale nie będzie to po prostu typowa korekta tekstu w trakcie
jego lektury. Za pomocą istniejących oraz stworzonych na potrzeby
projektu narzędzi weryfikowane będą indywidualnie wybrane aspekty
dygitalizacji. Oczekujemy, że dzięki temu weryfikacja będzie szybsza,
łatwiejsza i bardziej skuteczna.

% Na podstawie prawa autorskiego dygitalizacja w formie bazy danych
% kwalifikuje się do ochrony \textit{sui generis}; na mocy
% obowiązującego regulaminu nabywania, korzystania i ochrony własności
% intelektualnej na Uniwersytecie Warszawskim właścicielem baz danych
% jest uczelnia. Intencją jest udostępnianie jej użytkownikom w sposób
% swobodny, co wymaga uwzględnienia aspektów prawnych już na etapie
% formułowania wniosku o grant.

Z technicznego punktu widzenia możliwe jest udostępnianie zasobu przez
serwer baz danych (zapewne MySQL) dostępny zdalnie dla uprawnionych
użytkowników.


\section{Słownik Lindego}
\label{sec:slownik-lindego}

Uważam za celowe wskanowanie oryginałów obu wydań słowników (najlepiej
egzemplarzy z BUW) w sposób gwarantujący najwyższą jakość, tzn. za
pomocą ScanRobot dostępnego na zasadach komercyjnych w Pracowni
Skanowania przy Muzeum i Instytucie Zoologii PAN (\url{http://panscan.pl}).

Aktualnie wykorzystywany reprint sprawia wrażenie opracowywanego
komputerowo, co mogło wprowadzić jakieś zniekształcenia (przykładem
sygnalizowane wcześniej pseudo-oryginalne strony tytułowe tomu
szóstego).

Forma bazy danych pozwala w szczególności na równoległą analizę
różnych aspektów słownika, np. wykaz cytatów z poszczególnych źródeł
(por. \url{http://ebuw.uw.edu.pl/publication/339848}) czy indeksy słów dla
poszczególnych języków ---
por. \citep[s.~57]{ptaszyk07:_słown_samuel_bogum_lindeg}.

Zawartość bazy danych może być prezentowana użytkownikowi na różne
sposoby, np.
\begin{itemize}
\item warstwa tekstu ukrytego w dokumentach DjVu
\item korpus DjVu
\item korpus w formacie Poliqarp 2
\item HTML z zachowaniem układu strony oryginału lub z jego modyfikacją
\end{itemize}
W trakcie przygotowywania takiej formy prezentacyjnej wskazane jest
--- jeśli format wyjściowy na to pozwala --- scalenie erraty z tekstem
głównym,  wprowadzenia powiązań między oboma wydaniami słownika itp.

Na potrzeby interfejsu użytkownika może być potrzebne stworzenie
odpowiedniego fontu.

\section{Baza informacji typograficznych}
\label{sec:baza-inform-typogr-1}



Baza ta zawiera informacje wytworzone przez program rozpoznawania
znaków.  Podstawowe dane pobieramy z programu \p{tesseract} w formie
plików hOCR generowanego przez \p{ocrodjvu}. Podobne dane można
otrzymać z programu \p{tesseract} z łatą umożliwiającą wypisywanie
wyniku w formacie TSV
(por. \url{(http://tesseract-hocrtsv.herokuapp.com/},
\sout{\url{http://teksty.klf.uw.edu.pl/12/1/alice_1.png.hocr.tsv}}
\url{https://bitbucket.org/jsbien/linde-info/}); warto w miarę
możliwości zachować zgodnośc z formatem TSV.

Zakładamy segmentację do poziomu pojedynczych znaków. Jest ona
dostępna w plikach hOCR generowanych przez \p{ocrodjvu}, ale w
niestandardowej formie zaprojektowanej \textit{ad hoc} Przez Jakuba
Wilka --- znajduje się ona w komentarzu na końcu pliku
\sout{(por. np. \url{http://teksty.klf.uw.edu.pl/20/3/hOCR1.zip}}). W
bardziej czytelnej postaci jest dostępna w plikach djvused
generowanych przez \p{ocrodjvu}
\sout{(por. np. \url{http://teksty.klf.uw.edu.pl/20/4/djvused1.zip})} ---
jeśli dobrze pamiętam, \p{ocrodjvu} dwukrotnie uruchamia
\p{tesseracta} z różnymi parametrami i w ten sposób komasuje
wyniki. Być może warto w jakiś sposób wykorzystać faktt dostępnosci
danych w formacie djvused i kodu, który go generuje.

Dane mogą one być uzupełniane i konfrontowane z danymi z
innych podobnych programów (przede wszystkim \p{FineReader}). 

Punktem wyjścia jest zestaw własności przedstawiony niżej odnoszący
się do pojedynczej strony.

Niektóre wartości odnoszą się tylko do niektórych typów elementów; dla
pozostałych otrzymują one wartości puste lub ustalone umownie ---
np. własności fontu mogą być przypisane jednostkom większym niż znak,
jeśli wszystkie elementy podrzędne mają takie same ich wartości.

Szczególny charakter ma pole \texttt{text}. W sposób pierwotny jego
wartość jest wypełniana przez dane programu OCR, ale różne programy
--- a nawet różne przebiegi tego samego programu --- mogą dostarczać
tych wartości dla elementów różnych poziomów. Jeśli dysponujemy tą
informacją na poziomie znaków, to dodatkowo rekonstruujemy ją dla słów
i ewentualnie wierszy.

% JSB13_hOCRanaliza.tex



\begin{enumerate}
\item \f{version_id} (umowny odsyłacz do metadanych danej wersji,
  informujących o pliku wejściowym, użytym programie do OCR itp.; choć
  będziemy dążyć do tego, żeby istniała tylko jedna wersja bazy, do
  tej utworzenia potrzebne będą wersje alternatywne).
\item \f{page_id} (umowny identyfikator strony --- patrz niżej)
\item \f{level} (zagnieżdżenie elementu)
\item \f{page_id} (kolejny numer strony czyli elementu \texttt{div} z
  atrybutem \texttt{ocr_page}; informacja praktycznie chyba
  nieprzydatna, ale zachowana na wszelki wypadek);
\item \f{block_num} (kolejny numer bloku czyli elementu \texttt{div} z
  atrybutem \texttt{ocr_carea} na danej stronie --- jak w TSV, w hOCR numery są
  globalne! 0 dla level = 1);
\item \f{par_num} (kolejny numer akapitu czyli elementu \texttt{p} z
  atrybutem \texttt{ocr_par} w danym bloku --- jak w TSV, w hOCR numery są
  globalne! 0 dla level > 3);
\item \f{line_num} (kolejny numer wiersza czyli elementu \texttt{span} z
  atrybutem \texttt{ocr_line} w danym akapicie --- jak w TSV, w hOCR numery są
  globalne! 0 dla level > 4);
\item \f{word_num} (kolejny numer słowa czyli elementu \texttt{span} z
  atrybutem \texttt{ocrx_word} w danym wierszu --- jak w TSV, w hOCR numery są
  globalne! 0 dla level > 5);
\item \f{char_num} (kolejny numer znaku w słowie --- jeśli segmentacja na
  znaki nie jest dostępna w hOCR, to jest utworzona na podstawie
  \texttt{ocrx_word})
\item \f{left} (współrzędna gabarytu elementu, pobierana z sekcji
  \textsf{bbox} atrybutu \texttt{title} danego elementu; 0 tylko dla
  elementów, dla których hOCR nie zawiera ich gabarytów, czyli w
  praktyce znaki rozpoznane przez \p{FineReader})
\item \f{top} (współrzędna gabarytu elementu, pobierana z sekcji
  \textsf{bbox} atrybutu \texttt{title} danego elementu; 0 tylko jak wyżej)
\item \f{width} (rozmiar gabarytu elementu, pobierany z sekcji
  \textsf{bbox} atrybutu \texttt{title} danego elementu; 0 tylko jak wyżej)
\item \f{height} (rozmiar gabarytu elementu, pobierany z sekcji
  \textsf{bbox} atrybutu \texttt{title} danego elementu; 0 tylko jak wyżej)
\item \f{baseline} (pole tekstowe, zawierające zawartość sekcji
  \textsf{baseline} atrybutu \texttt{title} elementu \texttt{span}
  klasy \texttt{ocr_line}). Według specyfikacji
  \begin{quote}
    baseline pn pn-1 … p0 - a polynomial describing the baseline of a
    line of text
  \end{quote}
  Niestety aktualnie znaczenie wartości podawanych przez
  \textsf{tesseract} nie jest jasne. Są one wypisywane przez funkcję
  \texttt{AddBaselineCoordsTohOCR} w \texttt{api/baseapi.cpp}; w
  ogólnym wypadku może wystąpić również sekcja \texttt{textangle}. Dla
  elementów innych niż \texttt{ocr_line} wartość może być wyliczana
  według jakiegoś algorytmu; 0 jeśli hOCR nie podaje tej informacji
  dla \texttt{ocr_line}.
\item \f{conf} (ufność rozpoznania, pobierana z sekcji \textsf{x_conf}
  atrybutu \texttt{title} danego elementu; wartość 0 jeśli hOCR nie zawiera tej informacji)
\item \f{dir} (kierunek pisma, pobierany z atrybutu \texttt{dir} danego
  elementu; z wyjątkiem nielicznych wstawek w alfabecie hebrajskim
  zawsze \texttt{ltr} czyli \textit{left to right}). Wartość
  dziedziczona przez elementy podrzędne, oraz przez te elementy
  nadrzędne, dla których jest to jedyna wartość elementów podrzędnych;
  w przeciwnym razie 0. Wartość 0 również wtedy, gdy hOCR w ogóle nie
  zawiera tej informacji.
\item \f{lang} (język, pobierany z atrybutu \texttt{lang} danego
  elementu). Zasady dziedziczenia --- jak wyżej.
\item \f{font} (informacja pobierana z sekcji \textsf{x_font}
  atrybutu \texttt{title} danego elementu --- dostępna na razie tylko w
  wersji rozwojowej,
  \sout{(por. \url{http://teksty.klf.uw.edu.pl/13/})}. Zasady dziedziczenia --- jak wyżej.
% https://code.google.com/p/tesseract-ocr/issues/detail?id=1219&can=1&q=font%20info&sort=-i
%  Issue 1219: 	Add font info to hocr output

% https://code.google.com/p/tesseract-ocr/issues/detail?id=1219
% I extend it by config parameter hocr_font_info, so the user can turn it off (default ;-) ) or on..
% fixed in r1132.
\item \f{font_size}  (informacja o foncie, pobierana z sekcji \textsf{x_fsize}
  atrybutu \texttt{title} danego elementu --- dostępna na razie tylko w
  wersji rozwojowej
  \sout{(por. \url{http://teksty.klf.uw.edu.pl/13/})}. Zasady dziedziczenia --- jak wyżej.
\item \f{font_style} (eprezentowany przez element \texttt{strong}, element
  \texttt{em} lub brak tych elementów). Zasady dziedziczenia --- jak wyżej.
\item \f{text} (rozpoznany tekst w UTF-8 z wykorzystaniem Private Use Area
  dla znaków, słów i wierszy --- pobierany bezpośrednio z hOCR lub
  utworzony na podstawie elementów nadrzędnych lub podrzędnych).

\end{enumerate}

Wydaje się wskazane kilkakrotne przetwarzanie tekstu z różnymi
parametrami. Choć program \p{tesseract} może teoretycznie rozpoznawać
teksty wielojęzyczne, to zwiększanie liczby rozpoznawanych języków
pogarsza jakość rozpoznawania. W związku z tym planujemy zadawać
programowi jednocześnie co najwyżej dwa języki do rozpoznania. W
konsekwencji w bazie będziemy mieć kilka wersji danych typograficznych
i do dalszego przetwarzania będziemy wybierać wyniki najbardziej
prawdopodobne --- wydaje się to wygodniejsze, niż wstępnie
zaimplementowane przez Wilka komasowanie plików hOCR, które nie
radziłoby sobie z rozbieżnosciami w gabarytach. Trochę podobną metodologię
zastosował projekt LACE (\url{http://heml.mta.ca/lace}).

Dane typograficzne wymagają weryfikacji lub innej modyfikacji na
każdym poziomie, a wyniki weryfikacji powinne być zapisywane --- być
może w osobnej powiązanej bazie. Poniżej wymienione są przykładowe
możliwe testy.

\begin{enumerate}
\item nieuzasadnione zmiany linii pisma --- prawdopodobnie wynik
  traktowania skaz papieru jako znaków lub inny błąd segmentacji.
\item level (zagnieżdżenie elementu) --- zbyt duża wartość oznacza
  błąd rozpoznania.
\item block_num (kolejny numer bloku) --- należy oczekiwać błędów w
  rozpoznaniu podziału strony na łamy, potrzebne są też pewne decyzje
  konwencjonalne (paginę i sygnatury proponuję zaliczać do
  odpowiedniego łamu). Należy pamiętać o
  tabelach radiks i akapitach wspólnych dla obu łamów. Nie wszystkie
  bloki są prostokątne. Niewłaściwa liczba bloków sygnalizuje błąd
  rozpoznania.
% \item par_num (kolejny numer akapitu) --- może być błędny w
%   konsekwencji błędnego rozpoznania bloku.
% \item line_num --- może być błędny w konsekwencji błędnego rozpoznania
%   akapitu.
% \begin{quote}
%   Prawidłowe rozpoznanie wierszy jest niezbędne do poprawnej
%   linearyzacji tekstu (ustalenia kolejności czytania ---
%   \textit{reading order}) i stanowi warunek wstępny do dalszej
%   analizy, w szczególności podziału wierszy na wyrazy (chodzi przede
%   wszystkim o scalanie wyrazów przełamanych i analizy struktury haseł,
%   w szczególności ustalenia zakresu kwalifikatorów jezykowych).
% \end{quote}
% \item word_num (kolejny numer słowa) --- słowa są tu rozumiane
%   technicznie jako pewne napisy, mogące być w szczególności
%   pojedynczymi znakami. Numeracja ta nie jest jednak w pełni
%   wiarygodna, ponieważ znaki i napisy nierozpoznane przez program są
%   po prostu pomijane.
% \item char_num (kolejny numer znaku w słowie) --- należy się
%   spodziewać pewnej liczby błędów spowodowanych mylnym potraktowaniem
%   dwóch znaków jako jednego lub odwrotnie.
% \item left (współrzędna gabarytu elementu) --- wykorzystywana i
%   ewentualnie korygowana przez interakcyjny program do korekty.
% \item top (współrzędna gabarytu elementu) --- wykorzystywana i
%   ewentualnie korygowana przez interakcyjny program do korekty.
% \item width (rozmiar gabarytu elementu) --- wykorzystywany i
%   ewentualnie korygowana przez interakcyjny program do korekty.
% \item height (rozmiar gabarytu elementu) --- wykorzystywany i
%   ewentualnie korygowana przez interakcyjny program do korekty.
% \item baseline (pole tekstowe) --- nieprawidłowe wartości sygnalizują
%   wystąpienie przekosu lub inne poważne błędy skanowania.
% \item conf (ufność rozpoznania) --- obiekty o niskiej wartości powinne
%   być traktowane ze szczególną uwagą.
% \item dir (kierunek pisma) --- do sprawdzenia tylko w nielicznych
%   wstawkach w alfabecie hebrajskim.
% \item lang (język). Często wartość ta będzie oznaczać nie język
%   tekstu, a język użyty do rozpoznawania.
% \item font --- informacja istotna, ale w wynikach programu \p{tesseract} informacja
%   niewiarygodna, o ile nie zostanie on specjalnie wytrenowany na
%   potrzeby słownika. Informacja ta może jednak być pobrana z wyników
%   programu \p{FineReader}.
% \item font size --- nietypowy rozmiar fontu może wskazywać błąd
%   rozpoznania.
% \item font style --- informacja bardzo istotna dla analizy tekstu
%   przede wszystkim dlatego, że kursywą są podawane skróty nazw
%   cytowanych języków.
% \item text (rozpoznany tekst w UTF-8) --- niestety niektóre znaki będą
%   z pewnością rozpoznane błędnie, część z nich dlatego, że nie mają
%   swoich odpowiedników w standardzie Unicode.

\end{enumerate}

Strony identyfikujemy za pomocą globalnego numeru strony w tomie, z
uzwględnieniem następujacych reguł:
\begin{itemize}
\item w numeracji pomijamy tytularia przedruku,
\item w numeracji uwzgledniamy tytularia oryginału pominięte w
  przedruku,
\item tom szósty, podzielony na dwa wolumeny w przedruku
  GUTENBERG-PRINT, traktujemy jako jedną całość.
\end{itemize}

Pomocniczo będziemy korzystać z alternatywnej identyfikacja stron i
ich własności, które mogą wyglądać następująco:

\begin{enumerate}
\item Wydanie (reprint GUTENBERG-PRINT traktujemy jako wydanie czwarte
  --- pole może być pomijane, dopóki pracujemy tylko na jednym
  wydaniu).
\item Tom (1--6).
\item Część tomu (1-2 dla tomu szóstego, dla pozostałych 0 --- pole
  istotne tylko dla wydania czwartego).
\item Numer arkusza (ma znaczenie dla analizy układu strony ---
  wartość początkowa wpisana ręcznie, potem uzupełniana i weryfikowana
  automatycznie).
\item Numer strony w arkuszu (pierwsza zawiera sygnaturę główną,
  trzecia sygnaturę pomocniczą).
\item Numer sekcji paginacyjnej (zostaną zdefiniowane w osobnym
  dokumencie).
\item Typ paginacji (arabska, rzymska, ślepa, mieszana itp. ---
  zostaną zdefiniowane w osobnym dokumencie i wprowadzone ręcznie).
\item Numer strony w sekcji paginacyjnej.
\item Typ układu strony (np. hasłowa z tabelą, wakat - nietypowe
  wartości wpisane ręcznie, typowe uzupełnione i weryfikowane
  automatycznie).
\end{enumerate}



\section{Poziomy unilateralne}
\label{sec:poziomy-unilateralne}


Obiektami najniższego poziomu są obrazy czcionek na skanach. Mogą one
być niejednoznaczne z różnych powodów na jednym lub wielu
egzemplarzach dzieła: skazy papieru, wady druku (za mało lub za dużo
farby drukarskiej), defekt czcionki. W ogólnym wypadku nieoczywista
może też być segmentacja skanów na obrazy czcionek. Poziom ten
nazywamy typograficznym (z segmentacją) lub graficznym (bez
segmentacji).

Kolejnym poziomem są czcionki odtworzone na podstawie ich
obrazów. Wierne ich odtworzenie nie jest niezbędne dla dalszych etapów
dygitalizacji, ale może być przydatne np. do badania proweniencji
dzieła. Czcionki mają różnorodne własności --- wielkość, krój itp;
szczególna własnością jest wykaz ich użycia w tekście z dokładną
lokalizacją na stronie, zarówno względną (wiersz, numer), jak i
bezwględną (współrzędne w pikselach). Poziom ten nazywamy
faksymilowym.

Następnym poziomem są znaki kodowe w sensie Unicode; niektórym
czcionkom niezbędne jest arbitralne przypisanie kodów z obszaru użytku
prywatnego. Znaki wykorzystujemy jako wierne reprezentacje czcionek
--- na przykład \textit{vv} lub \textit{rv} w funkcji \textit{w} jest
reprezentowane jako \textit{vv} lub \textit{rv}. Poziom ten nazywamy
znakowym.

Wszystkie wymienione poziomy są unilateralne w sensie Saloniego, czyli
w żaden sposób nie nawiązują do znaczenia odpowiednich napisów.

Do opisu tych poziomów mogą być zaadoptowane narzędzia NPT (Ekstrakcja
i etykietowanie kształtów znaków --- shape extraction and labelling,
przeglądarka kształtów znaków --- a shape browser).

Warto podkreślić, że aktualnie nie jest znany pełny repertuar
występujacych w słowniku znaków. Najbardziej rzuca się w oczy
nietypowy znak paragrafu (por. \citep[s.~98]{bc347}), w cytatach
obcojęzycznych występuje długie s, można również spotkać np. wariant
znaku {\setmainfont{DejaVu Sans} ℟} (RESPONSE, U+211F). Kilka razy
występuje również {\setmainfont{DejaVu Sans} ⁊}(TIRONIAN SIGN ET,
U+204A) w kształcie \textit{r rotunda},
por. \url{https://en.wikipedia.org/wiki/Tironian_notes},
\url{https://bitbucket.org/jwilk/ocrodjvu/issue/15/}. Sporadycznie
występuje również znak {\setmainfont{DejaVu Sans} ☞} (WHITE RIGHT
POINTING INDEX, U+261E).

\section{Segmentacja unilateralna}
\label{sec:segm-unil}

Segmentacja unilateralna to segmentacja typograficzna czyli podział
obrazu strony na takie obszary, jak pagina, łamy, sygnatury, przypisy,
tabele oraz poszczególne wiersze tekstu. Podział ten w zasadzie daje
się przeprowadzić w sposób obiektywny, choć dostępne narzędzia nie
pozwalają go uzyskać w sposób w pełni automatyczny.

Do segmentacji typograficznej zaliczamy też segmentację wierszy przez
odstępy, która jednak nie jest kompletnie jednoznaczna --- z powodu
ciasnego składu nie zawsze można odróżnić odstępy międzywyrazowe od
międzyliterowych. Dlatego segmentację na tradycyjne wyrazy zaliczamy do
poziomów bilateralnych (patrz niżej).

Ważnym efektem ubocznym typograficznej segmentacji unilateralnej jest
opracowanie intuicyjnej konwencji lokalizacji fragmentu tekstu w
słowniku. \textbf{Wiersze są najmniejszą jednostką tekstu w~słowniku,
  które daje się jednoznacznie zidentyfikować na podstawie wyglądu
  strony}, dlatego należy przydzielić im identyfikatory, np. postaci
numeru tomu, numeru sekcji, numeru strony, identyfikatora bloku
(np. łamu) i lokalizacji wiersza w bloku (numer w bloku lub numer
akapitu i numer w akapicie).


\section{Poziomy bilateralne}
\label{sec:poziomy-bilateralne}

Pierwszym poziomem bilateralnym jest poziom teksteli (por. Bień 2011,
w druku). Na tym poziomie bierzemy pod uwagę funkcję czcionek, więc
\textit{vv} lub \textit{rv} w funkcji \textit{w} będzie reprezentowane
jako \textit{w}, a częsty w słowniku Lindego znak {\setmainfont{DejaVu
    Sans}\textit{z̓}} (LATIN SMALL LETTER Z, COMBINING COMMA ABOVE)
będzie reprezentowany przez \textit{ż}.

W przypadku słownika Lindego występuje systematycznie problem częstego
pomijania diakrytów w wersalikach, które nie zawsze mogą być
zrekonstruowane jednoznacznie --- por. \citep{bc379}. Na poziomie
tekstelowym powinne one być w razie potrzeby odtworzone.

Na tym poziomie nadal zachowane są ewentualne ligatury.

\section{Poziomy transkrypcyjne}
\label{sec:pozi-transkrypcyjne}

Poziomy transkrypcyjne opisują obiekty powstałe z segmentacji na słowa
strumienia teksteli uporządkowanych w kolejności czytania; oznacza to
sklejanie wyrazów przenoszonych do nowej linii, kolumny i strony z
pominięciem dywiza (niewykluczone są jednak przypadki, kiedy
interpretacja dywizu nie jest jednoznaczna). 

Jedną z naistotniejszych własności słowa jest język.

Do poziomów transkrypcyjnych zaliczamy w szczególności normalizację i
modernizację pisowni.

\printbibliography

\end{document}



%%% Local Variables:
%%% mode: latex
%%% TeX-master: t
%%% TeX-engine: xetex
%%% TeX-PDF-mode: t
%%% TeX-command-extra-options: "-shell-escape"
%%% coding: utf-8
%%% End:
